% -----------------------------------------------------------------------------
%   Arquivo: ./01-elementos-pre-textuais/resumoPt.tex
% -----------------------------------------------------------------------------



\begin{resumo}
Metaheurísticas híbridas reportam uma melhora significativa comparadas com as respectivas implementações clássicas. Apesar da técnica de hibridização de metaheurísticas ser promissora, ainda assim é possível encontrar problemas técnicos e fundamentalmente teóricos. As principais dificuldades técnicas são a necessidade do conhecimento do problema, e uma vez desenvolvida, é necessário calibrar os parâmetros numéricos do algoritmo. Uma alternativa ao desenvolvimento de novas meta-heurísticas, ou a hibridização manual delas, é utilizar os mecanismos de colaboração e comunicação próprios da modelagem de sistemas multi-agentes (MMAS) para promover uma hibridização dinâmica de diferentes estratégias de busca. A arquitetura D-Optimas é um MMAS baseado no modelo de atores, onde cada agente encapsula uma meta-heurística diferente e, dotado de um mecanismo de aprendizagem colabora com os demais agentes para encontrar a melhor solução para um problema de otimização. Os agentes interagem no espaço de busca que é dividido em regiões, que possuem um comportamento independente, podendo receber novas soluções, se particionar ou se fundir. O presente trabalho evoluiu a arquitetura D-Optimas comparada a sua última versão, adicionando mais algoritmos de otimização, atualizando a implementação para a biblioteca \textit{akka-cluster} e simplificando a sua execução em um \textit{cluster}. Este trabalho avaliou experimentalmente tanto a escalabilidade quanto o efeito da diversidade na qualidade das soluções. A arquitetura se mostrou escalável em um cluster de até seis nós, mantendo o número de agentes. A diversidade não se mostrou um fator relevante em todos os casos estudados.

\textbf{Palavras-chave}: Sistemas multi-agente; Otimização; Sistemas distribuídos.
 

\end{resumo}


\begin{resumo}[Abstract]
Hybrid metaheuristics report a significant improvement compared to their classical implementations. Although the hybridization technique of metaheuristics is promising, it is still possible to encounter technical and fundamentally theoretical problems. The main technical difficulties are the need to know the problem, and once developed, it is necessary to calibrate the numerical parameters of the algorithm. An alternative to the development of new meta-heuristics, or their manual hybridization, is to use the collaboration and communication mechanisms typical of multi-agent systems (MMAS) modeling to promote a dynamic hybridization of different search strategies. The D-Optimas architecture is an MMAS based on the actors model, where each agent encapsulates a different meta-heuristic and, equipped with a learning mechanism, collaborates with the other agents to find the best solution for an optimization problem. The agents interact in the search space, which is divided into regions, which have an independent behavior, being able to receive new solutions, partition or merge. The present work has evolved the D-Optimas architecture compared to its last version, adding more optimization algorithms, updating the implementation to the \textit{akka-cluster} library and simplifying its execution in a \textit{cluster}. This work experimentally evaluated both scalability and the effect of diversity on the quality of solutions. The architecture proved to be scalable in a cluster of up to six nodes, maintaining the number of agents. Diversity was not a relevant factor in all the cases studied.


\textbf{Keywords}: Multi-agent systems; Optimization; Distributed systems.
\end{resumo}