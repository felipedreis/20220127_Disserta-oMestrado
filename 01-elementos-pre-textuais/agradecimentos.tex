\begin{agradecimentos}
Não é possível começar este agradecimento sem antes trazer o contexto em que este trabalho se passou. Quando iniciei o curso do PPGMMC em 2019, jamais esperaria que o mundo fosse se deparar com uma pandemia no ano de 2020 que se estende até o presente momento. Presto aqui as minhas homenagens às tantas vítimas que esta doença fez, em especial no Brasil. Agradeço ao Eterno por ninguém próximo a mim ter se deparado com esta doença e padecido para ela, mas também rezo para que os que perderam entes queridos tenham o coração consolado.

Agradeço aos meus pais, que me incentivaram a estudar, e que jamais puseram limites até onde eu poderia ir. Espero sempre surpreendê-los e ir mais longe. 

Agradeço a minha noiva, Júlia, por todo afeto e compreensão, por todo apoio e energia. Por me socorrer nas horas mais difíceis. Por ter lido este texto incansáveis vezes. Juntos sonharemos sempre mais alto. 

Agradeço ao professor Henrique, que já se tornou a muito um querido amigo, com quem aprendo sempre. Obrigado pelo suporte ao longo deste trabalho, por todos os comentários cirúrgicos, pela gentil disposição em ensinar. Agradeço também ao professor Rogério, que também a muito tenho o privilégio de chamar de amigo, e que se juntou a este trabalho em um momento crítico em que tudo estava a ir pelos ares. Obrigado pela injeção de ânimo em cada reunião, ainda que demoradas, mas sempre com muitas risadas. 

Agradeço ao CEFET-MG, enquanto aluno do PPGMMC pelo suporte, e enquanto funcionário dele pela compreensão. Passei os últimos 13 anos da minha vida acadêmica nesta instituição, como aluno do técnico e da graduação, como funcionário e agora aluno da pós-graduação. Tenho imenso carinho pelo CEFET, que me abriu portas e mudou minha vida através da educação. Viva o ensino público de qualidade!

Agradeço aos amigos que o CEFET me deu, em especial Carolina Marcelino,  Gustavo Borba, Amanda Borba, Bárbara Jaber, Sinval Júnior, Vitor Peixoto, Lara Loures, Lucas Marioza, Paulo Hoffmam. Vocês acompanharam este trabalho de perto de alguma maneira, nos cafés, enquanto presencial, nas videochamadas durante o isolamento social. Obrigado por todo o suporte. 

Agradeço aos amigos que o Porto me deu, em especial Michelly, que chegou a tempo de testemunhar o desfecho deste trabalho. Obrigado, pelo alívio cômico e pela companhia nos almoços de domingo.

\end{agradecimentos}