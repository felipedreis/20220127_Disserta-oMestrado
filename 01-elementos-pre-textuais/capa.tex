% -----------------------------------------------------------------------------
%   Arquivo: ./01-elementos-pre-textuais/capa.tex
% -----------------------------------------------------------------------------



% -----------------------------------------------------------------------------
%   ATENÇÃO:
%   Caso algum campo não se aplique ao seu documento - por exemplo, em seu trabalho
%   não houve coorientador - não comente o campo, apenas deixe vazio, assim: \campo{}
% -----------------------------------------------------------------------------



% -----------------------------------------------------------------------------
%   Dados do trabalho acadêmico
% -----------------------------------------------------------------------------

\titulo{EFEITO DA HIBRIDIZAÇÃO DINÂMICA DE META-HEURÍSTICAS EM UM SISTEMA MULTIAGENTES DISTRIBUÍDO E ESCALÁVEL}

%\title{Title in English}
%\subtitulo{não sei se vai ter sub}
\autor{Felipe Duarte dos Reis}
\local{Belo Horizonte}
\data{Março de 2022}			% normalmente se usa apenas mês e ano



% -----------------------------------------------------------------------------
%   Natureza do trabalho acadêmico
%   Use apenas uma das opções: Tese (p/ Doutorado), Dissertação (p/ Mestrado) ou
%   Projeto de Qualificação (p/ Mestrado ou Doutorado), Trabalho de Conclusão de
%   Curso (Graduação)
% -----------------------------------------------------------------------------

\projeto{Projeto de Qualificação apresentado ao Programa de Pós Graduação em Modelagem Matemática e Computacional do Centro
Federal de Educação Tecnológica de Minas Gerais, como requisito parcial para obtenção do título de Mestre em Modelagem Matemática e Computacional}



% -----------------------------------------------------------------------------
%   Título acadêmico
%   Use apenas uma das opções:
%	Se a natureza for Tese, coloque Doutor
%	Se a natureza for Dissertação, coloque Mestre
%	Se a natureza for Projeto de Qualificação, coloque Mestre ou Doutor conforme o caso
%   Se a natureza for Trabalho de Conclusão de Curso, coloque Bacharel
% -----------------------------------------------------------------------------

%\tituloAcademico{Bacharel}



% -----------------------------------------------------------------------------
%   Área de concentração e linha de pesquisa
%	OBS: indique o nome da área de concentração e da linha de pesquisa do Programa de Pós-graduação
%   nas quais este trabalho se insere
%   Se a natureza for Trabalho de Conclusão de Curso, deixe ambos os campos vazios
% -----------------------------------------------------------------------------


\areaconcentracao{}
\linhapesquisa{}



% -----------------------------------------------------------------------------
%   Dados da instituicao
%   OBS: a logomarca da instituiçã deve ser colocada na mesma pasta que foi colocada o documento
%   principal com o nome de "logoInstituicao". O formato pode ser: pdf, jpf, eps
%   Se a natureza for Trabalho de Conclusão de Curso, coloque em "programa' o nome do curso de graduação
% -----------------------------------------------------------------------------

\instituicao{Centro Federal de Educação Tecnológica de Minas Gerais}
\programa{Programa de Pós-graduação em Modelagem Matemática e Computacional}
%\programa{Curso de Engenharia de Computação}
\logoinstituicao{0.2}{logoInstituicao}                  % \logoinstituicao{<escala>}{<nome do arquivo>}



% -----------------------------------------------------------------------------
%   Dados do(s) orientador(es)
% -----------------------------------------------------------------------------

\orientador{Henrique Elias Borges}
%\orientador[Orientadora:]{Nome da orientadora}
%\instOrientador{Centro Federal de Educação Tecnológica	de Minas Gerais -- CEFET-MG}

\coorientador{Rogério Martins Gomes}
%\coorientador[Coorientadora:]{Elizabeth Fialho Wanner}
%\instCoorientador{Centro Federal de Educação Tecnológica
%	de Minas Gerais -- CEFET-MG}


