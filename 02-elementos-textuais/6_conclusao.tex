\chapter{Conclusão}
\label{chap:conclusao}

\epigraph{Vivendo, se aprende; mas o que se aprende, mais, é só a fazer outras maiores perguntas.}{Riobaldo, em o \textit{Grande Sertão Veredas}}

Tendo sido apresentados os resultados dos experimentos acerca da diversidade no capítulo anterior, o presente capítulo retoma os objetivos iniciais deste trabalho, bem como a metodologia proposta. Inicialmente, o \autoref{chap:introducao} apresentou os fenômenos naturais que servem como inspiração tanto para a construção de sistemas multi-agentes como para metaheurísticas, introduzindo os conceitos de diversidade e colaboração. Após isso, são apresentados os elementos da literatura e problemas da vida prática que motivaram o desenvolvimento de novas metaheurísticas, bem como as limitações que essas estratégias tem em um fornecer boas soluções em diferentes famílias de problemas. Este argumento é construído em torno do \textit{"No Free Lunch Theorem"} para otimização \cite{wolpert1997}. Deste modo, é possível motivar e justificar a construção de um MAS, capaz de hibridizar de maneira dinâmica diferentes  metaheurísticas, de modo que elas consigam colaborar e aprender características de um determinado problema. Dado um sistema como este, é razoável esperar que uma configuração com uma variedade de estratégias de busca produza bons resultados para um número maior de problemas, em comparação com configurações pouco diversas. 

Na sequência, o \autoref{chap:revisao} visita a literatura de sistemas multi-agentes e \textit{frameworks} voltados para otimização, apresentando em especial aquelas que  possuem alguma característica distribuída. A maioria dos trabalhos encontrados na literatura são baseados em JADE (\textit{Java Agent DEvelopment Framework}), que dá suporte a programação distribuída. Entretanto, nenhum desses trabalhos apresentam resultados da solução de problemas de larga escala em um \textit{cluster}. O histórico da arquitetura D-Optimas é também apresentado neste capítulo, desde a sua primeira implementação, porposta por \citeonline{saliba2010}, até a sua última versão proposta por \citeonline{pacheco}, tendo sido completamente remodelada sobre o \textit{toolkit akka}.

Para permitir a execução da arquitetura em um número indeterminado de nós, algumas melhorias foram propostas no \autoref{chap:desenvolvimento}. O principal objetivo dessas melhorias eram prover um mecanismo de balanceamento de carga, transparência de localidade para as entidades da simulação e tolerância a falhas. Essas melhorias envolveram uma mudança na ontologia dos agentes, regiões e atores supervisores da simulação, bem como uma mudança nos protocolos de comunicação. Além disso, um novo mecanismo de memória, novas meta-heurísticas foram adicionadas e os algoritmos de particionamento e fusão das regiões foi aprimorado, de modo que ele não mais dependa de nenhum conhecimento prévio dos problemas.

Para avaliar os efeitos da nova organização dos atores na simulação o \autoref{chap:exp_preliminares} descreve o primeiro experimento  que comparou a média da latência de quatro tipos de mensagens na arquitetura, variando o número de nós no \textit{cluster}  de 3 à 6. A análise estatística permitiu dizer com 95\% de confiança que houve diferença nas latências sempre que dois ou mais nós eram adicionados à simulação. 

Finalmente, o \autoref{chap:exp_diversidade} apresentou os experimentos realizados para testar a principal hipótese deste trabalho, apresentada no \autoref{chap:metodologia}. Este experimentou simulou a arquitetura com uma diversidade de estratégias de busca, para verificar se este fator contribui para a produção de soluções de melhor qualidade, em comparação a execuções com um único agente explorador do espaço de busca. Foi necessário, para isso, prover um conjunto inicial de \textit{baseline}, com as configurações mais simples da arquitetura. Esta configuração básica foi sendo estendida a cada seção do capítulo, adicionando agentes de busca local, segmentando o espaço de busca, adicionando agentes do mesmo tipo com diferentes configurações e, finalmente, agentes de diferentes tipos.   

\section{Principais contribuições}

Retomando o objetivo principal apresentado no início deste trabalho, a principal meta aqui posta era consolidar a arquitetura D-Optimas do ponto de vista de um sistema distribuído. Esperava-se aprimorar os sistema, dando a ele algum grau de tolerância a falhas, por meio de um mecanismo de balanceamento de carga e transparência de localidade. Este tipo de modificação daria à arquitetura a possibilidade de ser executada em um \textit{cluster} com vários nós, escalando horizontalmente. 

Aberta a possibilidade de escalar o sistema horizontalmente, o próximo passo seria executar simulações com problemas mais complexos, com uma variedade de agentes e estratégias de busca. Deste modo seria possível observar a adaptação da arquitetura ao problema, com os agentes colaborando e, dessa forma, verificar, neste comportamento,  o surgimento de uma hibridização dinâmica das meta-heurísticas.

Para cumprir este objetivo geral, alguns objetivos específicos são listados no \autoref{chap:introducao}. O primeiro deles foi compreender a arquitetura D-Optimas, seus fundamentos teóricos e seus mecanismos de generalização de problemas. Uma revisão da literatura em sistemas multi-agentes, principalmente do histórico da arquitetura D-Optimas, desde a sua precursora Bimasco, foi apresentado no \autoref{chap:revisao}. O objetivo seguinte, que seria viabilizar a sua execução em um \textit{cluster} com um número arbitrário de nós, foi cumprido uma vez que uma nova versão da arquitetura foi proposta neste trabalho, baseada na biblioteca \textit{akka-cluster}. Foi proposta uma nova dinâmica para o comportamento das regiões e dos agentes no \autoref{chap:desenvolvimento}, que não dependam de um conhecimento específico do problema de otimização. Na sequência, também foram adicionadas novas meta-heurísticas populacionais, a saber, o algoritmo de evolução diferencial e o algoritmo de otimização por enxame de partículas. Estes dois pontos também estavam presentes na lista inicial de objetivos. 

Por fim, para estudar o comportamento da arquitetura, dois experimentos foram descritos no \autoref{chap:exp_preliminares} e \autoref{chap:exp_diversidade}. O primeiro experimento realizado avaliou a escalabilidade da arquitetura em um \textbf{cluster} em relação ao aumento do número de nós na rede. O  sistema se mostrou escalável para o problema escolhido, que foi a função \textit{EggHolder}, problema clássico da literatura. A métrica escolhida para avaliar a escalabilidade foi a latência média das mensagens trocadas pela arquitetura com maior frequência. Os resultados deram bons indícios da resiliência da arquitetura como um sistema distribuído, mas para obter resultados mais robustos é necessário avaliar outros fatores, como por exemplo, a complexidade do problema de otimização, o número máximo de regiões permitidas, o número de agentes em execução e o crescimento do conjunto de soluções produzido pela arquitetura. É necessário também avaliar o funcionamento da arquitetura num em um número maior de nós. 

O segundo experimento proposto avaliou o efeito da diversidade na qualidade das soluções produzidas pela arquitetura. Este experimento foi dividido em 5 etapas. A primeira delas serviu para obter um conjunto de \textit{baseline} de comparação, com a configuração mais simples possível da arquitetura. A segunda etapa avaliou a hibridização de um agente populacional e uma busca local em relação ao \textit{baseline}. Nesta etapa foi possível observar uma melhora dos resultados quando hibridizada a busca local ILS com o algoritmo populacional DE. As demais combinações não demostraram melhora estatística significativa. A terceira etapa avaliou o aumento do número de regiões para a configuração com um agente de busca local e um agente populacional. Neste caso, o aumento do número de regiões não fez diferença entre o experimento com uma única região para nenhuma das configurações propostas. A etapa seguinte avaliou a variação do número de agentes e suas respectivas configurações. Quando variado só o número de agentes, \textit{i.e.} aumentando a quantidade de agentes populacionais e busca local mas mantendo as configurações, os resultados mostraram que a arquitetura tem um desempenho melhor para a configuração com 10 regiões, mas este resultado não é extensível para um número maior de regiões. Também não é extensível para agentes iguais com diferentes configurações, experimento que para algumas configurações do número de regiões se mostrou inclusive pior do que o \textit{baseline}. A última etapa do procedimento experimental foi avaliar o desempenho da arquitetura com diferentes tipos de agentes em uma mesma simulação. Este resultado também não mostrou melhora em relação ao \textit{baseline}.

Os resultados alcançados no experimento a cerca da diversidade não permitem afirmar que a arquitetura exibe um resultado colaborativo de melhor qualidade na presença de diversidade de agentes. Entretanto, estes resultados também foram preliminares no sentido de que avaliaram a arquitetura em um único problema de otimização. Além disto, os resultados revelam que há dois problemas na versão atual da arquitetura. O primeiro é em relação à organização das regiões. Esperava-se que o número de regiões no sistema se estabiliza-se a medida que que novas soluções fossem produzidas, e que as regiões se organizassem de maneira a facilitar o aprendizado dos agentes a cerca das soluções mais interessantes. O observado é que as regiões raramente fazem fusão, e aumentam de maneira indiscriminada em direção ao limite de regiões do sistema. Aumentar o número de regiões também não surte efeito positivo para o agente, o que pode indicar um problema também na estratégia de seleção de ação implementada sobre a memória \textit{Q-learning}. 

De toda forma, a principal contribuição deste trabalho, para além da nova versão da arquitetura em um versão mais recente da biblioteca \textit{akka}, foi um procedimento experimental para avaliar tanto a escalabilidade quanto o efeito da diversidade na qualidade das soluções. Os trabalhos futuros serão discutidos com mais detalhes na próxima seção, mas é já está claro que estes experimentos aqui apresentados devem ser repetidos num conjunto maior de problemas, e avaliando mais fatores. Neste sentido, este trabalho proveu um roteiro experimental, com os devidos métodos estatísticos, que pode ser repetido simplesmente trocando o problema de otimização. A versão da arquitetura aqui apresentada também é mais robusta, no que diz respeito aos princípios de sistemas distribuídos, comparada com a versão anterior. A versão atual não tem um ponto único de falha em um ator coordenador, está baseada em um sistema de banco de dados distribuído, e utiliza um mecanismo de balanceamento de carga que redistribui as regiões pelos nós a medida que estas crescem ou diminuem. 

\section{Trabalhos futuros}

O presente trabalho evoluiu a arquitetura D-Optimas comparada a sua última versão, adicionando mais algoritmos de otimização, atualizando a implementação para a biblioteca \textit{akka-cluster} e simplificando a sua execução em um \textit{cluster}. Este trabalho também proveu todo o ferramental necessário para executá-la em diferentes infraestruturas, como o \textit{HPC} do CEFET-MG e a plataforma \textit{Google Cloud}. Como os resultados não indicaram que as modificações no sistema de memória e dinâmica das regiões melhorou de alguma forma o desempenho da arquitetura, estudar outras alternativas para a dinâmica das regiões, em especial a operação de fusão, parece um passo natural para um próximo trabalho. As regiões são a maneira que os agente segmentam e se comunicam dentro do espaço de busca, deste modo, um bom ajuste deste mecanismo é importante para o funcionamento da arquitetura.

Apesar das interações dos agentes com o mundo não ser determinística, num sentido de que não há uma ordem pré-estabelecida de qual agente vai trabalhar em qual conjunto de soluções, a sua interação com as regiões é limitada de modo que um agente só pode pedir soluções a uma região por vez. Adicionar mais comportamentos aos agentes traria a possibilidade de interações mais complexas com o ambiente, \textit{e.g.} o agente poderia solicitar soluções de diversas regiões ao mesmo tempo baseado em sua memória passada, diversificando a população em que ele trabalharia.

Por fim, um passo importante neste projeto seria avaliar a arquitetura o desempenho da arquitetura em um número maior de problemas, principalmente em problemas de diferentes categorias, \textit{e.g.} problemas de otimização combinatória, programação inteira, com várias restrições e multi-objetivos. Para isso, seria também interessante adicionar diferentes tipos de modificadores e estratégias dos algoritmos de busca local e populacional a fim de verificar a hipótese da diversidade em um conjunto maior de indivíduos. 
